\documentclass{article}
\usepackage{amsmath} % for \mid (divides)
\usepackage{amssymb} % for \square

\begin{document}

\section*{\small Paul Ujlaky}\vspace{-1em}
\section*{\small Discrete Mathematics}

\section*{Workshop 5 --- Revised Proof}

% Proving Equivalence Relation

\subsection*{1. Prove: $P(x) \, R \, P(x)$ (Reflexive)}

Let $P(x)$ be an arbitrary polynomial with arbitrary integer coefficients.

The relation $R$ states that $P(x) \, R \, Q(x)$ if and only if $n \mid (P(x) - Q(x))$ for an arbitrary integer $n$.

We know that $P(x) - P(x) = 0$. Similarly, we also know that $n \mid 0$ for any integer $n$. 

Because of this, it follows that $P(x) \, R \, P(x)$.

Therefore, $R$ is reflexive.

% Proving Symmetric Relation

\subsection*{2. Prove: $P(x) \, R \, Q(x) \implies Q(x) \, R \, P(x)$ (Symmetric)}

Let $P(x)$ and $Q(x)$ be arbitrary polynomials with arbitrary integer coefficients.

By the definition of $R$, we will assume $n \mid (P(x) - Q(x))$. This means $P(x) - Q(x) = nK(x)$ for some polynomial $K(x)$, due to the divides relationship.

Rearranging, 
\[
Q(x) - P(x) = -(P(x) - Q(x)) = -nK(x) = n(-K(x)).
\]

Since $-K(x)$ is a polynomial, it follows that $n \mid (Q(x) - P(x))$, so $Q(x) \, R \, P(x)$.

Therefore, $R$ is symmetric.

% Proving Transitive Relation

\subsection*{3. Prove: $P(x) \, R \, Q(x) \land Q(x) \, R \, Z(x) \implies P(x) \, R \, Z(x)$ (Transitive)}

Let $P(x)$, $Q(x)$, and $Z(x)$ be arbitrary polynomials with arbitrary integer coefficients.

We will assume $P(x) \, R \, Q(x)$ and $Q(x) \, R \, Z(x)$. By the definition of $R$:
\[
P(x) - Q(x) = nK(x), \quad Q(x) - Z(x) = nJ(x),
\]
for arbitrary polynomials $K(x)$ and $J(x)$.

Adding these equations:
\[
P(x) - Z(x) = (P(x) - Q(x)) + (Q(x) - Z(x)) = nK(x) + nJ(x) = n(K(x) + J(x)).
\]

Let $F(x) = K(x) + J(x)$. Since $F(x)$ is a polynomial, it follows that $n \mid (P(x) - Z(x))$.

Finally, it then follows that $P(x) \, R \, Z(x)$, and $R$ is transitive.

\subsection*{4. Conclusion}

Since $R$ is reflexive, symmetric, and transitive, it is an equivalence relation. 
\(\square\) % I had to do it

\end{document}